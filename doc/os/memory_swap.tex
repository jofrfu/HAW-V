\documentclass[a4paper, oneside]{scrreprt}
\usepackage[utf8]{inputenc}
\usepackage{graphicx}
\usepackage{listings}
\author{Andre Brand}

\begin{document}
\chapter{What is swappiness}

Swappiness is a factor for swapping pages from memory to the swap-partition on your hard-drive. Since Linux-Kernel 2.6 you are able to change the swappiness factor. The value can be between 0 and 100. If you choose 0 as a value, swapping will be disabled and you might run out of memory, because no pages from applications which are currently nit needed will stay in the memory.\\
If you choose the factor 100, every page of a freshly started program will instantly be swapped to the hard-drive and your computer might become really slow.\\
The default value is 60. At this value, pages will be swapped, when 40\% of the memory is used, so only 60\% is left free.

\chapter{Change the swappiness of Linux}

There are two ways to change the swappiness factor of Linux. The first one will change it imediately without a reboot, but it will be overwritten by the default value after a restart.
\begin{verbatim}
    sudo sysctl vm.swappiness=15
\end{verbatim}
This will change the swappiness factor to 15 so pages will be swapped to the hard-drive when 85\% of the memory is used.\\
\\
The other way will only take effect after a reboot, but it will change the swappiness factor constantly.\\
You have to change or add the line:
\begin{verbatim}
    vm.swappiness = 20
\end{verbatim}
in the file '/etc/sysctl.conf'.

\textbf{To see the swap in the linux kernel, look in the linux-folder named mm. There you can find the file swap.c and other c-files for memory swapping.}

\end{document}