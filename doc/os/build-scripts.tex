\documentclass[a4paper, oneside]{scrreprt}
\usepackage[utf8]{inputenc}
\usepackage{graphicx}
\usepackage{listings}
\author{Mike Wüstenberg}

\begin{document}
\chapter{Scripts}
 The purpose of this Document is to give a overview over the script included in the git repository.
 
\section{Table of Contents}
\begin{enumerate}
\item List of scripts 
\item build-toolchain 
\begin{enumerate}
\item Tour of the Sources
\item Using the tool-chain
\end{enumerate}
\item{Troubleshooting}
\end{enumerate}

\section{List of scripts}
\begin{itemize}
\item build-toolchain.sh, builds the riscv-tools.
\item build-linux.sh, builds the riscv-linux.
\end{itemize}

\chapter{build-toolchain}
The build of the 64bit tool-chain can be split in 4 parts. First we need to install 3rd party tools like "git" and "curl". After successful installation we can download the git repository and update the sub-modules.
\begin{lstlisting}
git clone https://github.com/riscv/riscv-tools
git submodule update --init --recursive
\end{lstlisting}
Warning this will require a several gigabyte of space. Now we can build the the tool-chain with the build.sh script included in the riscv-tools repository.
The last step is to test the tool-chain. The script writes a small C which should print "Your Toolchain is working!" on your Terminal.

\section{Tour of the Sources}
After the execute of the script the folder should look like this.
\begin{itemize}
\item build-toolchain.sh, the script.
\item riscv-tools, the repository clone.
\item riscv, the built tool-chain.
\item riscv/toolchaintest, location of the test program.
\item riscv/bin, location of the compiler and the spike emulator.
\end{itemize}
\section{Using the tool-chain}
You compile a simple c program with with one of the gcc compiler located in "riscv/bin" for example.
\begin{lstlisting}
riscv/bin/riscv64-unknown-elf-gcc -o YOUR-PROGRAM YOUR-PROGRAM.c
\end{lstlisting} 
The tool-chain includes a risc-v emulator named spike to run C a program. The command is "spike pk YOUR-PROGRAM". The option "pk" provides a simple Proxy Kernel for your program to run in.
\begin{lstlisting}
riscv/bin/spike pk YOUR-PROGRAM
\end{lstlisting}
For easy access you can add the bin folder as a path variable. "PATH=\$PATH:riscv/bin"

\chapter{build-linux}
This script will build the a 64bit riscv-linux. This alone will not boot you will need a file system for that. 
For the building process you will need the riscv-tools, and the 3rd party tools. 
With the the 3rd party tool we will download the riscv-linux repository and a Linux kernel.
\begin{lstlisting}
git clone https://github.com/riscv/riscv-linux.git
\end{lstlisting}
After that we can configure the Linux makefile and build the image.


\chapter{Troubleshooting}
\section{Functions not found}
It can happen that the make file dose not find some functions. If that happens you can try to set the cross compile variable.
\begin{lstlisting}
make -j4 ARCH=riscv CROSS_COMPILE=riscv64-unknown-linux-gnu-
\end{lstlisting}

\end{document}