\documentclass[a4paper, oneside]{scrreprt}
\usepackage[utf8]{inputenc}
\usepackage{graphicx}
\usepackage{listings}
\author{Andre Brand}

\begin{document}
\chapter{Inline Assembler}
To write an inline assembler program which works on spike, you just need to write your instructions as strings in an asm()-block in a c-file.
\\
\section{Example}
\begin{lstlisting}
int main(int argc, char* argv) {
	asm(
		// Marker, to find the start of this assembler code
		"li t1, 0xaffedead;"
		"li t1, 0x3;"
		"add t2, t1, t1;"
		"addi t2, t2, 0x3;"
		"end: j end;"
	);
}
\end{lstlisting}

\chapter{Running Inline Assembler in Spike}
To run your inline assembler code in spike, you have to compile it with the gcc included in the risc-v toolchain.\\
\begin{lstlisting}
riscv64-unknown-elf-gcc -o yourprogramname yourcodefile.c
\end{lstlisting}
Then you have to run the compiled program in spike using the debug-mode and the proxykernel.\\
\begin{lstlisting}
spike -d pk yourprogramname
\end{lstlisting}
After the assembler program has started you can use the enter-key to do a single step. When you type run 10 spike will step over the next 10 instructions. When you just type run, your program will be runned until it's finished.
\\
To get to the start of your program you need to type:
\begin{lstlisting}
until reg 0 t1 0xaffedead
\end{lstlisting}

This command will run your code until the register t1 from core with index 0 contains the value "0xaffedead".
\\
To show the values of registers you need to type:
\begin{lstlisting}
reg 0 t1
\end{lstlisting}

This command will show you the value of register t1 from the core with index 0.
\end{document}