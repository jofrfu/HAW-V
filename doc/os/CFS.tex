% Author	:	Talal Tabia

\documentclass[a4paper, oneside]{scrreprt}
\usepackage[utf8]{inputenc}
\usepackage{listings}
\usepackage{hyperref}

\begin{document}

\chapter{Completely Fair Scheduler (CFS)}

\section{Characteristics}

CFS is the linux scheduler and it has:
\begin{itemize}
	\item no heuristics
	\item good I/O Handling
	\item cpu bound processors
\end{itemize}

\section{Task}
Fair and equal distribution of CPU time among the processes\\
Ideal Fairness:
Each process in the ready queue gets (100/n) percent of the CPU time.

In each cycle - time slice in miliseconds. 
The processor time is equal: n / x
The scheduler tries to distribure the CPU time among \\the processes in each cycle

The whole thing is done using Virtual Runtimes. 
Every executale process has the vruntime.
-- At each scheduling point, if the process was started \\for t miliseconds, 
the vruntime = vruntime + t

\section{CFS idea}
When a timer interrupt occurs:
\begin{itemize}
\item Select the task with the smallest vruntime (min vruntime) - Pointer
\item The process is then executed in the CPU
  \item If another interrupt occurs, 
then context change to a task with a smaller runtime
\end{itemize}

For the administration of different tasks and runtime, \\the CFS uses red-black-tree:
\begin{itemize}
\item Each node represents a runnable task
\item The nodes are sorted by their vruntime
\item Node left is smaller than node right
\item The node that is most left has the min vruntime
\end{itemize}

- The scheduler takes the task with min vruntime to execute next.

We wrote in this project a small programm in python \\that works with that, it change cfs parameters \\and shows the current processor usage of single cpus.

The location of the programm is in the folder: \\ach-ne-2017-2018/src/os/linuxscheduler/
And the programm main file is: LinuxSchedulerWindows.py

\end{document}
