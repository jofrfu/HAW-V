\documentclass[a4paper, oneside]{scrreprt}
\usepackage[utf8]{inputenc}
\usepackage{graphicx}
\usepackage{listings}
\author{Andre Brand}

\begin{document}
\chapter{Write inline Assembler Code}

Please check, that the RISC-V toolchain is already installed on your system!\\
\\
To write inline Assembler code for the RISC-V emulator spike, you have to write a C-File. Ever Assembler Command must be between two " Signs and have to end with a ';' and all Commands must be in an asm()-block.\\
When you start your code with the ProxyKernel which is included by spike, you have to put a unique value in a register at the beginning of the program.\\
Here is an example:\\
\begin{verbatim}
int main() {
    asm(
        "li a1, 0xaffedead;"
        "li a1, 3;"
        "add a2, a1, a1;"
        "addi a2, a2, 1;"
        "li a4, 2;"
        "mul a3, a1, a4;"
        "mul a3, a3, a4;"
        "end: j end;"
    );
    return 0;
}
\end{verbatim}

After that compile your written program with the command:
\begin{verbatim}
    riscv64-unknown-elf-gcc [-o <programm name>] <filename>
\end{verbatim}

\chapter{Start your code with Spike-Emulator}

If you have compiled your program, you can start it with the command:
\begin{verbatim}
    spike -d pk <programm name>
\end{verbatim}

After that you can jump to the begining of your program with the command:
\begin{verbatim}
    until reg 0 <register> <unique value>
\end{verbatim}
With this command, you are able to jump to the line, where the value of the given register is equal to your unique value.\\
\\
Now you are able to ahve a look at all registers of the cpu by typing:
\begin{verbatim}
    reg 0 <register>
\end{verbatim}
By hitting the Enter-key you can go a single step forward.
All other commands can be displayed when you type help. Here is the list, which will be shown:
\begin{verbatim}
    Interactive commands:
    reg <core> [reg]                # Display [reg] (all if omitted) in <core>
    fregs <core> <reg>              # Display single precision <reg> in <core>
    fregd <core> <reg>              # Display double precision <reg> in <core>
    pc <core>                       # Show current PC in <core>
    mem <hex addr>                  # Show contents of physical memory
    str <hex addr>                  # Show NUL-terminated C string
    until reg <core> <reg> <val>    # Stop when <reg> in <core> hits <val>
    until pc <core> <val>           # Stop when PC in <core> hits <val>
    until mem <addr> <val>          # Stop when memory <addr> becomes <val>
    while reg <core> <reg> <val>    # Run while <reg> in <core> is <val>
    while pc <core> <val>           # Run while PC in <core> is <val>
    while mem <addr> <val>          # Run while memory <addr> is <val>
    run [count]                     # Resume noisy execution 
                                      (until CTRL+C, or [count] insns)
    r [count]                         Alias for run
    rs [count]                      # Resume silent execution 
                                      (until CTRL+C, or [count] insns)
    quit                            # End the simulation
    q                                 Alias for quit
    help                            # This screen!
    h                                 Alias for help
    Note: Hitting enter is the same as: run 1
\end{verbatim}

\end{document}