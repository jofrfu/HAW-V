% File		: 	main.tex
% Created by:	Sebastian Brückner
% Created at :	03.10.17 13:00
% Authors	:	Andre Brand

\chapter{Quick guide for Plan B}

For Plan B we are using an existing implementation for an RISC-V architecture with a linux distribution. 
This is find in the following repository. 
https://github.com/ucb-bar/fpga-zynq 

\section{Step one}
Clone the repository.

\section{Step two}
Switch to zedboard folder in terminal and run the command make fetch-images.
\section{Step three}
Switch to the fpga-images-zedboard folder and copy \textit{boot.bin, devicetree.dtb, uImage and uramdisk.image.gz} to a \textbf{fap32} formatted sd-card.
\section{Step four}
Set the Jumper Position of JP11-JP7 to the following position to set as SD-BOOT mode:

\begin{tabular}{|c|c|} \hline
 JP11 & 0 \\ \hline
 JP10 & 1 \\ \hline
 JP9  & 1 \\ \hline
 JP8  & 0 \\ \hline
 JP7  & 0 \\ \hline
\end{tabular}

Connect the micro-usb to the uart connector and power up the zedboard.
\section{Step five}
Run the following command:

\textit{screen /dev/tty.usbmodem1411115200,cs8,-parenb,-cstopb}

The linux should power on on the ARM Processor.
The \underline{username} and the \underline{password} are \textbf{root}.

\section{Step six}
Run the following command to start a linux on the RISC-V processor:

\textit{./fesvr-zynq bbl}
