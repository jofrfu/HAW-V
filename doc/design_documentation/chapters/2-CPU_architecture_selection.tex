% File		: 	main.tex
% Created by:	Sebastian Brückner
% Created at :	02.10.17 19:00
% Authors	:	Sebastian Brückner

\chapter{CPU Architecture Selection}

\section{RISC vs CISC}

The CPU in this project is an RISC CPU, because
\begin{itemize}
	\item a CISC CPU might be to complex to be easily understood
	\item modern CPU tend to be RISC CPUs
	\item a CISC might be to complex to implement within the scope of this project 
\end{itemize}

\section{Instruction sets}
There are plenty of choices when it comes to instruction sets, 
this list compiles the most obvious.

\begin{itemize}
	\item  ARM
	\item Thumb-2
	\item AVR
	\item OpenRisc
	\item RISC-V
\end{itemize}

While ARM (and Thumb-2) are well documented, the ARM is not a RISC by the book.
Also there a plenty of ARM devices around and this project aims to break fresh ground.

AVR is not really suited for an CPU that is supposed to run an OS. 
Its better suited for small microcontrollers.

Also these architectures are nor open source.
This is the main advantage of OpenRISC and RISC-V.

RISC-V was chosen over OpenRISC because:

\begin{itemize}
	\item a very good documentation
	\item modular Instruction Set Architecture with multiple well defined extensions
	\item existing compiler (gcc toolchain)
	\item existing QEMU simulator to develop the OS on
	\item not many existing CPU designs (being capable to break fresh ground)
	\item the RISC-V Foundation has strong backers (e.g. Google and IBM) \footnote{https://riscv.org/membership/}
\end{itemize}

 