% File		: 	5-OS_riscv_poky.tex
% Created by:	Mike Wüstenberg
% Created at :	13.02.2018
% Authors	:	MIke Wüstenberg

\chapter{Riscv Poky}

\section{Poky}

Riscv-poky is a port of the Yocto project. It's a full Linux distribution which enables user to 
cross-compile and package software as file system.

\subsection{Yocto project}

The open source project Yocto provides tools, templates and methods to create Linux based systems. 
The package creation is based on recipe. A recipe describes how to cross-compile and fetch a package.
This allows to build recipes for a wide range of build targets. The complete Linux images is build with
the BitBake tool.

\subsection{BitBake}

BitBake is a efficient execution engine which works in parallel and executes Python tasks.
The instructions on what tasks BitBake should run are stored in recipe(.bb), configuration(.conf) and 
class(.bbclass) files.
\\
The recipes includes:
\begin{itemize}
\item Description
\item Recipe version
\item Dependencies
\item Where to get the Source code
\item patch for the source code
\item Compile instructions
\item install location
\end{itemize}
A list of all BitBake recipe versions can be shown with the "bitbake -s" command.
\\ \\
The Configuration file defines various configurations. This includes user configuration options, 
compiler tuning options and more. The Class files contains information which are useful the to 
share between files. For more information take a look at the Yocto documentation.

\begin{itemize}
\item Yocto Project: http://www.yoctoproject.org/docs/current/ref-manual/ref-manual.html
\item BitBake: http://www.yoctoproject.org/docs/1.6/bitbake-user-manual/bitbake-user-manual.html
\end{itemize}

\section{Quickstart}

First donwload the riscv-poky repository and initialize the OpenEmbedded environment.

\begin{verbatim}
git clone https://github.com/riscv/riscv-poky.git
cd riscv-poky
source oe-init-build-env
\end{verbatim}

This takes you to the build directory. The "conf/local.conf" file is preconfigured to build for riscv64. 
In case you want to use a already installed tool-chain you can change the configuration. Once the
compilation has finished you will find is inside the "tmp" folder. To start the build process run 
the following command.

\begin{verbatim}
bitbake core-image-riscv
\end{verbatim}
This will build a a riscv-v image which includes Python, Perl, SSH and other tools. Since this will
download a large amount of data this can take several hours. But you can on completion run a fully
operational Linux. Test your Linux with the following command.

\begin{verbatim}
runspike riscv64
\end{verbatim}

\section{Adding recipes}
To add a recipe to your image you need to change the image .bb file. The file can be found in the following 
path.

\begin{verbatim}
/riscv-porky/meta-riscv/recipes-core/image
\end{verbatim}
The default image has the name "core-image-riscv.bb". Copy the default which creates a basic Linux.
Now go edit the new file. You can add new recipes with.
\begin{verbatim}
IMAGE_INSTALL += "git"
\end{verbatim}
As long as there is a recipe for the package you want to install this will install it on your new image.
Build your Linux image with the "bitbake" command and the new image recipe you just created. BitBake will use
previous build software to speed up the process.

\section{Troubleshooting}
\subsection{Failed to fetch URL}
 In case a commit got orphaned and is not "visible" from any branch you can add nobranch=1; add to the fetch command.
 \\
 Source: https://github.com/riscv/riscv-poky/issues/15

\begin{verbatim}
WARNING: qemu-native-2.5.0-r0 do_fetch: Failed to fetch URL gitsm://github.com/riscv/riscv-qemu.git;destsuffix=/home/user/RISC-V/riscv-poky/build/tmp/work/x86_64-linux/qemu-native/2.5.0-r0/qemu-2.5.0, attempting MIRRORS if available
ERROR: qemu-native-2.5.0-r0 do_fetch: Fetcher failure: Unable to find revision 9bfcd4717b3010eb7efc50057232e92ecb741cac in branch master even from upstream
ERROR: qemu-native-2.5.0-r0 do_fetch: Fetcher failure for URL: 'gitsm://github.com/riscv/riscv-qemu.git;destsuffix=/home/user/RISC-V/riscv-poky/build/tmp/work/x86_64-linux/qemu-native/2.5.0-r0/qemu-2.5.0'. Unable to fetch URL from any source.
ERROR: qemu-native-2.5.0-r0 do_fetch: Function failed: base_do_fetch
ERROR: Logfile of failure stored in: /home/user/RISC-V/riscv-poky/build/tmp/work/x86_64-linux/qemu-native/2.5.0-r0/temp/log.do_fetch.1244
ERROR: Task (virtual:native:/home/user/RISC-V/riscv-poky/build/../meta-riscv/recipes-devtools/qemu/qemu_2.5.0.bb:do_fetch) failed with exit code '1'
\end{verbatim}
\begin{verbatim}
In the file meta-riscv/recipes-devtools/qemu/qemu_2.5.0.bb replace SRC_URI as follows:
SRC_URI = "git://github.com/riscv/riscv-qemu.git;nobranch=1;destsuffix=${S}
\end{verbatim}





